\documentclass{beamer}
\usetheme{Berkeley}
\usecolortheme{sidebartab}
\begin {document}
\title{GROW IT YOURSELF}   
\author{Guided by: Prof. Vidyavati N. Deshpande\\Team leader: Miss.Priyanka S. Tibile\\Member: Miss. Gayatri M. Jangam\\Member:Miss. Asiya A.Desai\\Member:Miss. Trupti C. Patil\\} 
\date{\today} 

\frame{\titlepage} 

\frame{\frametitle{Table of contents}\tableofcontents} 


\section{Introduction} 
\frame{\frametitle{Introduction} 
Aeroponics is the process of growing plants in an air or mist environment without the use of soil or an aggregate medium. The word aeroponics is derived from the Greek meanings of aer and ponos.\\It is the practice of growing plants without soil, with roots in a misty environment.}
\subsection{Abstract}
\frame{ 
Ardunio is gaining popularity day by day. \\Right from Industrial Automation to Software to manufacturing, Ardunio is making its way.\\ However the agricultural practices used even today are far way from the deployment for the benefit.\\ People still follow the obsolete agricultural practices.\\The crop plantation requires lot of hard work for a farmer while factors such as soil fertility,water and many such environmental conditions and additional to them the crop diseases will affect the larger percent of agricultural produce for most of the farmers. \\While some nutrient values may vary which creates a major impact on crops.\\To find a proper solutions to these particular problems this project is created.}


\section{Methodology} 
\frame{\frametitle{Methodology}
	\begin{itemize}
		\item Grow LED light
		\item Ultrasonic Atomizer (fog maker)
		\item ph sensing
		\item Air Control System 
	\end{itemize}
}
\subsection{Hardware used}
\frame{\frametitle{Hardware used}
\begin{itemize}
\item pH sensor 
\item Ardunio Mega 2560
\item UV Sensor
\item Temperature sensor
\item Light intensity sensor
\item UV and normal leds
\item Relay
\item Ultrasonic sensor
\item 2.4 inch touch display
\item DHT11 Humidity sensor
\item Power supply
\end{itemize} 
}

\subsection{Software used}
\frame{\frametitle{SOFTWARES}
\begin{enumerate}
\item Ardunio IDE
\item Serial monitor
\end{enumerate}
}
\section{Advantages} 
\frame{\frametitle{Advantages}
	\begin{itemize}
		\item No Soil 
		\item Easy to automatize 
		\item Controlled environment
		\item Shorter grow cycles 
		\item Vertical farming possible
		\item Less water
		\item Indoor
		\item Air quality enhancement
	\end{itemize}
}
\section{Disadvantages} 
\frame{\frametitle{Disadvantages}
	\begin{itemize}
		\item Power Dependency in Aeroponics System
		\item Initial setup for Aeroponics System
		\item Technical Knowledge Is Required		
	\end{itemize}
}

\frame{\frametitle{Applications}
	\begin{itemize}
		\item Retail /hotel/fast food Chains
		\item Private Investors
		\item Public Sector Companies
		\item Railway Catering companies(IRCTC)
		\item NASA
		\item Corporate Hospitals
		\item Fresh Fruits and Vegetables Exporters  
		\item Large land owners
		\item NGO’s
		\item Foreign Retail Companies 
	\end{itemize}
}

\section{Conclusion} 
\frame{\frametitle{Conclusion}
		\begin{itemize}
		\item We did a lot of research and analysed carefully the strengths and weaknesses of the available projects, and considering the market response,came up with a new advanced open source system, which will reduced the disadvantages and will implement only the best features.
		\item The Commercial Aeroponics industry is a successful industry and is rapidly expanding. 
		\item The market is larger than opined as produce is sold on quality rather than production method. 
		\item Aeroponics cannot displace bulk commodity items.
		\item The industry is expected to grow exponentially as conditions of soil growing is becoming difficult. 
		\item Government intervention and university interest can propel the use of this technology.
		
	\end{itemize}
}
\end{document}